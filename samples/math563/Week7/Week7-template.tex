\documentclass[12pt]{gradescopeexam}
% \documentclass[12pt,noanswers,solutionkey]{gradescopeexam}
\noprintanswers
% \printanswers
% \printsolutionkey


\usepackage{amsmath}
\usepackage{amsthm}
\usepackage{bbm}
\usepackage{amsthm}
\usepackage{amsfonts}
\usepackage{times}
\usepackage{graphicx}
\usepackage{multicol}
\usepackage{amsthm}
\usepackage{datetime}
\usepackage{hyperref}
\usepackage{comment}
\usepackage{listings}
\usepackage{xcolor}
\usepackage{units}
\usepackage{MnSymbol}
\usepackage{pgfplots}

\renewcommand\P{\mathbb{P}}
\newcommand\R{\mathbb{R}}
\newcommand\Q{\mathbb{Q}}
\newcommand\E{\mathbb{E}}
\newcommand\N{\mathbb{N}}
\newcommand\F{\mathbb{F}}
\newcommand\one{\mathbbm{1}}
\newcommand\V{\mathrm{var}}
\newcommand\sgn{\mathrm{sgn}}
\renewcommand\vec[1]{\mathbf{#1}}
\renewcommand\c[1]{\mathcal{#1}}
\newcommand\im[1]{\mathrm{Im}(#1)}

\newtheorem*{lemma}{Lemma}

\begin{document}

% ----------------------------------------------------------------
\LeftBoxMargin{0.0\linewidth}
\RightBoxMargin{0.0\linewidth}
\AssignmentTitle{Week7-template}
\CourseName{Math 563, Fall 2022}
% \FirstName{Marek}
% \LastName{Rychlik}
% \StudentId{31415926}
% ----------------------------------------------------------------
\makeheader
\vspace{0.1in}
\begin{questions}
  \begin{question}
    (Durrett 4.1.1.) \textbf{Bayes's Formula.} Let $G\in\c{G}$.
    \begin{parts}
      \part Show that
      \[ \P(G \mid A) = \frac{\int_G \P(A\mid \c{G})\,d\P}{\int_\Omega\P(A\mid \c{G})\,d\P}\]
      \begin{prooforbox}[6.3in]
      \end{prooforbox}
      \part Show that when $\c{G}$ is generated by a partition $\{G_1,G_2,\ldots\}$, this
      reduces to the usual Bayes' formula:
      \[ \P(G\mid A) = \frac{\P(A \mid G_j)\,\P(G_j)}{\sum_j \P(A \mid G_j)\, \P(G_j)}. \]
      \begin{prooforbox}[6.5in]
      \end{prooforbox}
    \end{parts}
  \end{question}

  \begin{question}
    (Durrett 4.1.2.) Prove \textbf{Chebyshev's inequality.} If $a>0$ then
    \[ \P(|X|\ge q \mid \c{F}) \le a^{-2} \E(X^2 \mid \c{F}) \]
    \begin{prooforbox}[6.5in]
    \end{prooforbox}
  \end{question}

  \begin{question}
    (Durrett 4.1.5.) Give an example on $\Omega=\{a,b,c\}$ in which:
    \[ \E(\E(X \mid \c{F}_1)|\c{F}_2) \neq \E(\E(X \mid \c{F}_2)\mid \c{F}_1)  \]
    \begin{solutionorbox}[6.5in]
    \end{solutionorbox}
  \end{question}

  \begin{question}
    (Durrett 4.1.9.) Show that if $X$ and $Y$ are random variables with $\E(Y\mid \c{G})=X$ and
    $\E(Y^2)=\E(X^2)<\infty$, then $X=Y$ a.s.
    \begin{solutionorbox}[6.5in]
    \end{solutionorbox}
  \end{question}

  \begin{question}
    (Durrett 4.1.10.) \textbf{\color{magenta} Bonus problem!}  The
    result of the last exercise implies that if $\E Y^2 <\infty$ and
    $\E(Y \mid \c{G})$ has the same distribution as $Y$ then
    $\E(Y \mid \c{G})=Y$ a.s.  Prove that under the assumption
    $\E |Y| <\infty$. Hint: The trick is to prove that
    $\sgn(X)=\sgn(\E(X \mid \c{G}))$ a.s., and then take $X=Y-c$ to get
    the desired result.
    \begin{solutionorbox}[6.5in]
    \end{solutionorbox}
  \end{question}

  
\end{questions}
\end{document}

%%% Local Variables:
%%% mode: latex
%%% TeX-master: t
%%% TeX-engine: default
%%% End:
