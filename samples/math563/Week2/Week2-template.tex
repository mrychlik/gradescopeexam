\documentclass[12pt]{gradescopeexam}
% \documentclass[12pt,noanswers,solutionkey]{gradescopeexam}
\noprintanswers
% \printanswers
% \printsolutionkey


\usepackage{amsmath}
\usepackage{bbm}
\usepackage{amsthm}
\usepackage{amsfonts}
\usepackage{times}
\usepackage{graphicx}
\usepackage{multicol}
\usepackage{amsthm}
\usepackage{datetime}
\usepackage{hyperref}
\usepackage{comment}
\usepackage{listings}
\usepackage{xcolor}
\usepackage{units}

\renewcommand\P{\mathbb{P}}
\newcommand\R{\mathbb{R}}
\newcommand\Q{\mathbb{Q}}
\newcommand\E{\mathbb{E}}
\newcommand\one{\mathbbm{1}}
\newcommand\V{\mathrm{var}}
\renewcommand\vec[1]{\mathbf{#1}}
\renewcommand\c[1]{\mathcal{#1}}

\begin{document}

% ----------------------------------------------------------------
\LeftBoxMargin{0.0\linewidth}
\RightBoxMargin{0.0\linewidth}
\AssignmentTitle{Week2-template}
\CourseName{Math 563, Fall 2022}
% \FirstName{Marek}
% \LastName{Rychlik}
% \StudentId{31415926}
% ----------------------------------------------------------------
\makeheader
\vspace{0.1in}
\begin{questions}
  \begin{question}
    (Durrett 1.2.1) Suppose $X$ and $Y$ are random variables on
    $(\Omega, \c{F}, \P )$ and let $A \in \c{F}$.  Show that if we let
    $Z(\omega) = X(\omega)$ for $\omega \in A$ and
    $Z(\omega) = Y (\omega)$ for $\omega \in A^c$ , then $Z$ is a random
    variable.
    \begin{parts}
      \part Complete the equation by replacing $\square$ with the correct formula:
      \[ Z^{-1}(B) = \left(A\cap(\square)\right) \cup \left(A^c\cap (\square) \right)\]
      \begin{solutionorbox}[1.5in]
      \end{solutionorbox}
      \part Complete the proof.
      \begin{prooforbox}[3in]

      \end{prooforbox}
    \end{parts}
  \end{question}

  \begin{question}
    (Durrett 1.2.2) Let $\chi$ have the standard normal
    distribution. Use Theorem 1.2.6 to get upper and lower bounds on
    $\P (\chi \ge 4)$.
    \begin{parts}
      \part What is the upper bound?
      \begin{solutionorbox}[3in]

      \end{solutionorbox}
      \part What is the lower bound?
      \begin{solutionorbox}[3in]

      \end{solutionorbox}
    \end{parts}
  \end{question}

  \begin{question}
    (Durrett 1.2.3)
    Show that a distribution function has at most countably many
    discontinuities.
    \begin{prooforbox}[6in]

    \end{prooforbox}
  \end{question}
  \begin{question}
    (Durrett 1.2.4) Show that if $F (x) = \P (X \le x)$ is continuous
    then $Y = F (X)$ has a uniform distribution on $(0,1)$, that is,
    if $y \in [0, 1]$, $P (Y \le y) = y$.
    \begin{parts}
      \part Prove the conclusion with an additional assumption that
      $F$ is strictly increasing.
      \begin{prooforbox}[3in]

      \end{prooforbox}
      \part Carry out the proof (without this additional assumption).
      \begin{prooforbox}[3.5in]

      \end{prooforbox}
    \end{parts}
  \end{question}

  \begin{question}
    (Durrett 1.2.5) Suppose $X$ has continuous density $f$ ,
    $\P (\alpha \le X \le \beta) = 1$ and $g$ is a function that is
    strictly increasing and differentiable on $(\alpha, \beta$).  Then
    \begin{parts}
      \part $g(X)$ has density $f (g^{-1}(y))/g'(g^{-1} (y))$ for
      $y \in (g(\alpha), g(\beta))$ and $0$ otherwise.
      \begin{prooforbox}[3.5in]
        
      \end{prooforbox}
      \part When $g(x) = ax + b$ with $a > 0$,
      $g^{-1} (y) = (y - b)/a$ so the answer is $(1/a)f ((y - b)/a)$.
      \begin{prooforbox}[3in]

      \end{prooforbox}
    \end{parts}
  \end{question}

  \begin{question}
    (Durrett 1.2.6) Suppose $X$ has a normal distribution. Use the
    previous exercise to compute the density of $\exp(X)$. (The answer
    is called the lognormal distribution.)
    \begin{prooforbox}[7.5in]

    \end{prooforbox}
  \end{question}
  \begin{question}
    (Durrett 1.2.7)
    \begin{parts}
      \part (i) Suppose $X$ has density function $f$.  Compute
      the distribution function of $X^2$ and then differentiate to find
      its density function.
      \begin{prooforbox}[3in]

      \end{prooforbox}
      \part (ii) Work out the answer when $X$ has a standard normal
      distribution to find the density of the chi-square distribution.
      \begin{prooforbox}[3in]

      \end{prooforbox}
    \end{parts}
  \end{question}

  \begin{question} (Custom Problem, generalizes Durrett 1.2.5) Suppose
    $X$ has continuous, density $f$ , and $g$ is a continuously
    differentiable function (not necessarily increasing) such that
    that the set $$\{y\in\R: g'(y) = 0\}$$ has Lebesgue measure
    zero.

    \begin{parts}
      \part 
      Let $Y=g(X)$. Prove that $Y$ has density $h$ given by the equation
      \[ h(x) = \sum_{y\in g^{-1}(x)} \frac{f(y)}{|g'(y)|}. \] 
      \begin{prooforbox}[6in]

      \end{prooforbox}
      \part Let $g:(0,1]\to(0,1]$ be the \emph{fractional part} of $1/x$, i.e.
      function defined by
      \[ g(x) = \frac{1}{x} -\left\lfloor \frac{1}{x}\right\rfloor \]
      Let $X$ be a continuous random variable whose distribution has
      probability density function (pdf):
      \[ f(x) =
        \begin{cases}
          \frac{1}{\pi}\frac{1}{1+x}, & 0 \le x \le 1,\\
          0 \quad\text{otherwise.}
        \end{cases}
      \]
      Prove that $Y=g(X)$ is also a continuous variable with the same pdf $f$.
      \begin{prooforbox}[6in]

      \end{prooforbox}
    \end{parts}
  \end{question}
\end{questions}
\end{document}

%%% Local Variables:
%%% mode: latex
%%% TeX-master: t
%%% TeX-engine: default
%%% End:
