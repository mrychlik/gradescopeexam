\documentclass[12pt]{gradescopeexam}
% \documentclass[12pt,noanswers,solutionkey]{gradescopeexam}
\noprintanswers
%\printanswers
%\printsolutionkey


\usepackage{amsmath}
\usepackage{bbm}
\usepackage{amsthm}
\usepackage{amsfonts}
\usepackage{times}
\usepackage{graphicx}
\usepackage{multicol}
\usepackage{amsthm}
\usepackage{datetime}
\usepackage{hyperref}
\usepackage{comment}
\usepackage{listings}
\usepackage{xcolor}
\usepackage{units}
\usepackage{MnSymbol}

\renewcommand\P{\mathbb{P}}
\newcommand\R{\mathbb{R}}
\newcommand\N{\mathbb{N}}
\newcommand\Q{\mathbb{Q}}
\newcommand\E{\mathbb{E}}
\newcommand\one{\mathbbm{1}}
\newcommand\V{\mathrm{var}}
\newcommand\cF{\mathcal{F}}     %sigma-algebra
\newcommand\cS{\mathcal{S}}     %for space of intervals
\newcommand\cR{\mathcal{R}}     %for Borel subsets.
\newcommand\cC{\mathcal{C}}     %for Borel subsets.
\newcommand\cA{\mathcal{A}}     %for Borel subsets.
\renewcommand\vec[1]{\mathbf{#1}}

\begin{document}

% ----------------------------------------------------------------
\LeftBoxMargin{0.0\linewidth}
\RightBoxMargin{0.0\linewidth}
\AssignmentTitle{Week1-template}
\CourseName{Math 563, Fall 2022}
%\FirstName{Marek}
%\LastName{Rychlik}
%\StudentId{31415926}
% ----------------------------------------------------------------
\makeheader
\vspace{0.1in}
\begin{questions}
  \begin{question} (Analysis, Set Theory Review)
    \begin{parts}
      \part Carefully state the \textbf{Axiom of Choice}. Use proper notation,
      consistent with the book.
      \begin{solutionorbox}[6in]
        
      \end{solutionorbox}
      \newpage
      \part 
      Let $x_i$, $i\in I$, be an uncountable family of positive real numbers.
      Let $|J|$ denote the cardinality of a subset $J\subseteq I$.
      Prove that
      $$\sup_{J\subset I,\; |J|<\infty} \sum_{i\in J} x_i = \infty.$$
      Make sure to show the step which relies upon the Axiom of Choice.
      \begin{prooforbox}[6in]
      \end{prooforbox}
    \end{parts}
  \end{question}

  \newpage
  \begin{question} (Durrett, Problem 1.1.1)
    Let $\Omega=\R$, $\cF =$  all subsets so that $A$ or $A^c$ is countable,
    $\P(A)=0$ in the first case and $=1$ in the second. Show that $(\Omega,\cF,\P)$
    is a probability space.
    \begin{parts}
      \part Which are the \textbf{axioms} of a $\sigma$-algebra? (When in doubt, follow the book.)
      \begin{choices}
        \choice closed under complement
        \choice closed under finite intersection
        \choice closed under countable union
        \choice contains empty set $\emptyset$
        \choice contains $\Omega$
        \choice contains all subsets of $\Omega$        
      \end{choices}
      \newpage
      \part Prove that $\cF$ is a $\sigma$-algebra.
      \begin{prooforbox}[6in]
      \end{prooforbox}
      \newpage
      \part Which are \textbf{properties} (including the axioms) of $\P$, a general probability measure? (When in doubt, follow the book.)
      \begin{choices}
        \choice The domain of $\P$ is $2^\Omega$.
        \choice The domain of $\P$ is $\cF$.
        \choice $\P(A\cup B) = \P(A)+\P(B)$ for $A,B\in\cF$.
        \choice $\P(A^c)=\P(A)$
        \choice $\P(A^c)=1-\P(A)$        
        \choice $\P(\emptyset) = 0$.
        \choice If $\P(A)=0$ then $A=\emptyset$.
        \choice If $A\subseteq B$ then $\P(A) \le \P(B)$ for $A,B\in\cF$.
      \end{choices}
      \newpage
      \part Prove that $\P$ defined in 1.1.1 is a probability measure.
      \begin{prooforbox}[6in]
      \end{prooforbox}
    \end{parts}
  \end{question}
  \begin{question}
    (Durret 1.1.2)
    Let $\cS_d=$ the empty set plus all
    sets of the form
    $$(a_1,b_1]\times\ldots(a_d,b_d]\subset \R^d$$
    where $-\infty \le a_i < b_i\le\infty$.
    Show that $\sigma(\cS_d) = \cR^d$, the Borel subsets of $\R^d$.
    \begin{prooforbox}[7in]
    \end{prooforbox}
  \end{question}
  \begin{question}
    (Durret 1.1.3)
    A $\sigma$-field $\cF$ is said to be countably generated if there is a
    countable collection $\cC \subseteq \cF$ so that $\sigma(\cC) = \cF$. Show that $\cR^d$
    is countably  generated.
    \begin{prooforbox}[7.5in]
    \end{prooforbox}
  \end{question}
  \begin{question}
    (Durret 1.1.4)
    \begin{parts}
      \part (i) Show that if $\cF_1 \subseteq \cF_2 \subseteq \ldots$ are $\sigma$-algebras,
      then $\bigcup_i\cF_i$ is an algebra.
      \begin{prooforbox}[7.5in]

      \end{prooforbox}
      \newpage
      \part (ii) Give an example to show that $\bigcup_i\cF_i$ need not be a $\sigma$-algebra.
      \begin{solutionorbox}[7.5in]
      \end{solutionorbox}
    \end{parts}
  \end{question}
  
  \begin{question}
    (Durrett 1.1.5) A set $A \subseteq \{1, 2, \ldots\}$ is said to
    have asymptotic density $\theta$ if
    $$\lim_{n\to\infty} |A \cap \{1, 2, \ldots , n\}|/n = \theta$$
    Let $\cA$ be the collection of sets for which the asymptotic density exists.
    \begin{parts}
      \part Is $\cA$ a $\sigma$-algebra?
      \begin{choices}
        \choice Yes
        \choice No
      \end{choices}
      \part   an algebra?
      \begin{choices}
        \choice Yes
        \choice No
      \end{choices}
      \newpage
      \part Justify the answers in previous parts.
      \begin{prooforbox}[7.5in]
      \end{prooforbox}
    \end{parts}
  \end{question}
\end{questions}
\end{document}
