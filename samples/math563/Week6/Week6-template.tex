\documentclass[12pt]{gradescopeexam}
% \documentclass[12pt,noanswers,solutionkey]{gradescopeexam}
\noprintanswers
% \printanswers
% \printsolutionkey


\usepackage{amsmath}
\usepackage{amsthm}
\usepackage{bbm}
\usepackage{amsthm}
\usepackage{amsfonts}
\usepackage{times}
\usepackage{graphicx}
\usepackage{multicol}
\usepackage{amsthm}
\usepackage{datetime}
\usepackage{hyperref}
\usepackage{comment}
\usepackage{listings}
\usepackage{xcolor}
\usepackage{units}
\usepackage{MnSymbol}
\usepackage{pgfplots}

\renewcommand\P{\mathbb{P}}
\newcommand\R{\mathbb{R}}
\newcommand\Q{\mathbb{Q}}
\newcommand\E{\mathbb{E}}
\newcommand\N{\mathbb{N}}
\newcommand\F{\mathbb{F}}
\newcommand\one{\mathbbm{1}}
\newcommand\V{\mathrm{var}}
\renewcommand\vec[1]{\mathbf{#1}}
\renewcommand\c[1]{\mathcal{#1}}
\newcommand\im[1]{\mathrm{Im}(#1)}

\newtheorem*{lemma}{Lemma}

\begin{document}

% ----------------------------------------------------------------
\LeftBoxMargin{0.0\linewidth}
\RightBoxMargin{0.0\linewidth}
\AssignmentTitle{Week6-template}
\CourseName{Math 563, Fall 2022}
% \FirstName{Marek}
% \LastName{Rychlik}
% \StudentId{31415926}
% ----------------------------------------------------------------
\makeheader
\vspace{0.1in}
\begin{questions}
  \begin{question}
    (Durrett 3.1.1.)
    Generalize the proof of Lemma 3.1.1 to conclude that if
    $\max_{1\le j \le n} |cj,n | \to 0$, $\sum_{j=1}^n c_{j,}\to\lambda$
    and $\sup_{n}\sum_{j=1}^n|c_{j,n}|<\infty$ then $\prod_{j=1}^n(1+c_{j,n})\to e^{\lambda}$.
    \begin{prooforbox}[6.5in]

    \end{prooforbox}
  \end{question}

  \begin{question}
    (Durrett 3.1.2.)
    If the $X_i$ have a Poisson distribution with mean 1, then $S_n$ has
    a Poisson distribution with mean $n$, i.e., $\P(S_n = k) = e^{-n} n^k /k!$.
    \begin{parts}
      \part 
      Use Stirling’s formula to show that if $(k - n)/ n \to x$ then
      \begin{equation*}
        \sqrt{2\pi\,n}\, \P (S_n = k) \to \exp(-x^2/2).
      \end{equation*}
      \begin{prooforbox}[6.5in]

      \end{prooforbox}
      \part
      Use the asymptotic expansion for $\log n!$
      (\href{https://en.wikipedia.org/wiki/Stirling%27s_approximation}{see
        \emph{Wikipedia} page} for details):
      \begin{align*}
        \ln(k!) - \tfrac{1}{2}\ln k
        & = \tfrac{1}{2}\ln 1 + \ln 2 + \ln 3 + \cdots + \ln(k-1) + \tfrac{1}{2}\ln k\\
        & = k \ln k - k + 1 + \sum_{\ell=2}^{\infty} \frac{(-1)^\ell B_\ell}{\ell(\ell-1)} \left( \frac{1}{k^{\ell-1}} - 1 \right) 
      \end{align*}
      where  $B_\ell$ is a Bernoulli number, to find the asymptotic expansion of
      $\P (S_n = k)$ in $n$. Negative powers of $n$ are to be expected. 
      \begin{solutionorbox}[5.5in]
      \end{solutionorbox}
    \end{parts}
  \end{question}

  \begin{question}
    (Durrett 3.1.4.)
    Suppose $\P(X_i = k) = e^{-1} /k!$ for $k = 0, 1, \ldots$. Show that if $a > 1$
    $$\frac{1}{n}\log\P(S_n \ge n\,a) \to a - 1 - a\,\log a.$$
    \begin{prooforbox}[6.5 in]
    \end{prooforbox}
  \end{question}


  \begin{question}
    (Custom) Use the method of generating functions to calculate the
    \textbf{the exact} probability that in 6 independent die rolls we get a total score of 15.
    This problem may seem to involve a lot of computations, but it is not so if the
    method of generating functions is used efficiently.
    \begin{parts}
      \part Write the probability generating function for a single die roll score $X_1$
      (uniforml probabilty of $1/6$ for die faces $\Omega=\{1,2,\ldots,6\}$
      \begin{solutionorbox}[3in]
      \end{solutionorbox}
      \part Simplify the formula to ratio of polynomials with two terms only. (HINT: Use the
      formula for the sum of a finite geometric series \[ \sum_{k=0}^n x^k = (1-x^{n+1})/(1-x)
      =(1-x^{n+1})(1-x)^{-1}.\]
      \begin{solutionorbox}[2.5in]
      \end{solutionorbox}
      \part Write the probability generating function for the sum $S_n=\sum_{j=1}^n X_j$, where
      each $X_j$ represents the face value of a singl die roll.
      \begin{solutionorbox}[2in]
      \end{solutionorbox}
      \part Use the Cauchy product formula and the binomial expansion for any real $\alpha$:
      \[ (1+x)^{\alpha} = \sum_{k=1}^\infty {\alpha \choose k} x^k \]
      where
      \[ {\alpha\choose k} = \frac{\alpha(\alpha-1)\cdots(\alpha-k+1)}{k!} \]
      to write down the formula for the probability
      $P(S_n = k)$. The formula should involve a single summation. Work out the limits.
      They will involve functions such as $\min$, $\max$, $\lfloor\cdot\rfloor$ and
      $\lceil \cdot \rceil$. Also, you may find the following identity useful,
      which allows rewriting binomial coefficients for negative $\alpha$ with a positive one:
      \[ {-n \choose k} = (-1)^k { n + k - 1 \choose k} \]
      \begin{solutionorbox}[2.5in]
      \end{solutionorbox}
      \part
      Evauluate the previously derived formula for $P(X_n=k)$ with $n=6$ and
      $k=15$ ``by hand'' to get the exact value.
      \begin{solutionorbox}[5in]
      \end{solutionorbox}
    \end{parts}
  \end{question}


\end{questions}
\end{document}

%%% Local Variables:
%%% mode: latex
%%% TeX-master: t
%%% TeX-engine: default
%%% End:
