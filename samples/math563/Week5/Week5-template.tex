\documentclass[12pt]{gradescopeexam}
% \documentclass[12pt,noanswers,solutionkey]{gradescopeexam}
\noprintanswers
% \printanswers
% \printsolutionkey


\usepackage{amsmath}
\usepackage{amsthm}
\usepackage{bbm}
\usepackage{amsthm}
\usepackage{amsfonts}
\usepackage{times}
\usepackage{graphicx}
\usepackage{multicol}
\usepackage{amsthm}
\usepackage{datetime}
\usepackage{hyperref}
\usepackage{comment}
\usepackage{listings}
\usepackage{xcolor}
\usepackage{units}
\usepackage{MnSymbol}
\usepackage{pgfplots}

\renewcommand\P{\mathbb{P}}
\newcommand\R{\mathbb{R}}
\newcommand\Q{\mathbb{Q}}
\newcommand\E{\mathbb{E}}
\newcommand\N{\mathbb{N}}
\newcommand\F{\mathbb{F}}
\newcommand\one{\mathbbm{1}}
\newcommand\V{\mathrm{var}}
\renewcommand\vec[1]{\mathbf{#1}}
\renewcommand\c[1]{\mathcal{#1}}
\newcommand\im[1]{\mathrm{Im}(#1)}

\newtheorem*{lemma}{Lemma}

\begin{document}

% ----------------------------------------------------------------
\LeftBoxMargin{0.0\linewidth}
\RightBoxMargin{0.0\linewidth}
\AssignmentTitle{Week5-template}
\CourseName{Math 563, Fall 2022}
% \FirstName{Marek}
% \LastName{Rychlik}
% \StudentId{31415926}
% ----------------------------------------------------------------
\makeheader
\vspace{0.1in}
\begin{questions}
  \begin{question}
    (Durrett 2.3.6.) Metric for convergence in probability. Show
    \begin{parts}
      \part (a) that $d(X, Y ) =\E\left(\frac{|X - Y |}{1 + |X - Y |}\right)$
      defines a metric on the set of random variables,
      i.e.,
      \begin{enumerate}
      \item $d(X, Y ) = 0$ if and only if $X = Y$ a.s.,
      \item $d(X, Y ) = d(Y, X)$,
      \item $d(X, Z) \le d(X, Y ) + d(Y, Z)$
      \end{enumerate}
      \begin{prooforbox}[2.5in]
      \end{prooforbox}
      \part and (b) that $d(X_n , X) \to 0$ as $n\to\infty$
      if and only if $X_n \to X$ in probability.
      \begin{prooforbox}[2.5in]
      \end{prooforbox}
    \end{parts}
  \end{question}

  \begin{question}
    (Durrett 2.3.10.) Kochen-Stone lemma. Suppose $\sum_k \P (A_k ) = \infty$. Use Exercises
    1.6.6 and 2.3.1 to show that if
    \[ \limsup_{n\to\infty}\frac{\left(\sum_{k=1}^n \P(A_k)\right)^2}%
      {\sum_{1\le j,k\le n} \P(A_j\cap A_k)} = \alpha > 0 \]

    then $\P (A_n \text{i.o.}) \ge \alpha$. The case $\alpha = 1$ contains Theorem 2.3.7.
    \begin{prooforbox}[6in]
    \end{prooforbox}
  \end{question}
  \begin{question}
    (Durrett 2.3.14.) Let $X_1, X_2 , \ldots$ be independent. Show that $\sup X_n < \infty$ a.s. if
    and only if $\sum_n \P (X_n > A) < \infty$ for some $A$.
    \begin{prooforbox}[4in]
    \end{prooforbox}
  \end{question}

  \begin{question}
    (Durrett 2.3.15.) Let $X_1 , X_2 , \ldots$ be i.i.d. with $\P (X_i > x) = e^{-x}$, let $M_n =
    \max_{1\le m\le n} X_m$ . Show that
    \begin{parts}
      \part (i) $\limsup_{n\to\infty} X_n / \log n = 1$ a.s.
      \begin{prooforbox}[3in]
      \end{prooforbox}
      \part and (ii) $M_n / \log n \to 1$ a.s.
      \begin{prooforbox}[3in]
      \end{prooforbox}
    \end{parts}
  \end{question}

  \begin{question}
    (Durrett 2.3.17.) Let $Y_1 , Y_2 , \ldots$ be i.i.d.
    Find necessary and sufficient conditions
    for
    \begin{parts}
      \part (i) $Y_n /n \to 0$ almost surely,
      \begin{solutionorbox}[3.4in]
      \end{solutionorbox}
      
      \part (ii) $\max_{m\le n} Y_m /n \to 0$ almost surely,
      \begin{solutionorbox}[3.4in]
      \end{solutionorbox}
      
      \part (iii) $\max_{m\le n} Y_m /n \to 0$ in probability,
      \begin{solutionorbox}[3.5in]
      \end{solutionorbox}
      
      \part and (iv) $Y_n /n \to 0$ in probability.
      \begin{solutionorbox}[3.5in]
      \end{solutionorbox}
    \end{parts}
  \end{question}  
\end{questions}
\end{document}

%%% Local Variables:
%%% mode: latex
%%% TeX-master: t
%%% TeX-engine: default
%%% End:
