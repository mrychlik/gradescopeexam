\documentclass[12pt]{gradescopeexam}
% \documentclass[12pt,noanswers,solutionkey]{gradescopeexam}
\noprintanswers
% \printanswers
% \printsolutionkey


\usepackage{amsmath}
\usepackage{bbm}
\usepackage{amsthm}
\usepackage{amsfonts}
\usepackage{times}
\usepackage{graphicx}
\usepackage{multicol}
\usepackage{amsthm}
\usepackage{datetime}
\usepackage{hyperref}
\usepackage{comment}
\usepackage{listings}
\usepackage{xcolor}
\usepackage{units}
\usepackage{MnSymbol}

\renewcommand\P{\mathbb{P}}
\newcommand\R{\mathbb{R}}
\newcommand\Q{\mathbb{Q}}
\newcommand\E{\mathbb{E}}
\newcommand\one{\mathbbm{1}}
\newcommand\V{\mathrm{var}}
\renewcommand\vec[1]{\mathbf{#1}}
\renewcommand\c[1]{\mathcal{#1}}


\begin{document}

% ----------------------------------------------------------------
\LeftBoxMargin{0.0\linewidth}
\RightBoxMargin{0.0\linewidth}
\AssignmentTitle{Week3-template}
\CourseName{Math 563, Fall 2022}
% \FirstName{Marek}
% \LastName{Rychlik}
% \StudentId{31415926}
% ----------------------------------------------------------------
\makeheader
\vspace{0.1in}
\begin{questions}
  \begin{question}
    (Durrett 1.3.2.) Prove Theorem 1.3.6 when $n = 2$ by checking $\{X_1 +X_2 < x\} \in \c{F}$.
    \begin{parts}
      \part Let $X = X_1+ X_2$. Prove that we have a partition
      \begin{equation}
        \label{eqn:preimage-decomp}
        X^{-1}((-\infty,x]) = \bigcupdot_{u\in\R} X_1^{-1}(u) \cap X_2^{-1}((-\infty,x-u]).
      \end{equation}
      \begin{prooforbox}[3in]
      \end{prooforbox}
      \part Why is the fact in the previous part not sufficient to
      deduce that $X^{-1}((-\infty,x])\in\c{F}$?
      \begin{solutionorbox}[2.3in]
      \end{solutionorbox}
      \part If $X_1$ is a discrete variable, use the partition in
      equation~\eqref{eqn:preimage-decomp} to show that $\{X<x\}\in\c{F}$.
      \begin{prooforbox}[3in]
      \end{prooforbox}
      \part Write $X^{-1}((-\infty,x])$ as a triple-nested expression
      (e.g. countable union of countable intersections of countable unions) like the partition above,
      which shows by inspection that $\{X<x\}\in\c{F}$.
      \begin{solutionorbox}[4in]
      \end{solutionorbox}
    \end{parts}
  \end{question}

  \begin{question}
    (Durrett 1.3.5.) A function $f$ is said to be lower semicontinuous or l.s.c. if
    \[ \liminf_{y\to x} f (y) \ge f (x) \]
    and upper semicontinuous (u.s.c.) if $-f$ is l.s.c. Show that $f$ is l.s.c. if
    and only if $\{x : f (x) \le a\}$ is closed for each $a \in \R$ and conclude that
    semicontinuous functions are measurable.
    \begin{prooforbox}[6.5in]
    \end{prooforbox}
  \end{question}
  
  \begin{question}
    (Durrett 1.4.4.) Prove the Riemann-Lebesgue lemma. If $g$ is
    integrable (in the sense of Lebesgue) then
    \[ \lim_{n\to \infty}   \int_{-\infty}^\infty g(x) \cos n\,x\,dx = 0. \]
    Hint: If g is a step function, this is easy. Now use the previous exercise (Durrett 1.4.3).
    NOTE: Use the results of that exercise freely but do not solve it.
    \begin{prooforbox}[6.5in]
    \end{prooforbox}
  \end{question}

  \begin{question}
    (Durrett 1.5.8.) Show that if f is integrable on $[a, b]$,
    $g(x) = \int_{[a,x]} f(y)\, dy$ is continuous on $(a, b)$.
    \begin{prooforbox}[7.5in]
    \end{prooforbox}
  \end{question}

  \begin{question}
    (Durrett 1.6.14.) Let $X \ge 0$ but do NOT assume $\E(1/X) < \infty$. The book defines
    this notation
    \[ \E(X; A) = \int_{A} X\,d\P = \E(X\cdot \one_A)\]
    Show
    \begin{parts}
      \part
      \[ \lim_{y\to\infty} y\,\E(1/X; X > y) = 0,\]
      \begin{prooforbox}[2.8in]
      \end{prooforbox}
      \part 
      \[\lim_{y\downarrow 0} y\,\E(1/X; X > y) = 0. \]
      \begin{prooforbox}[2.6in]
      \end{prooforbox}
    \end{parts}
  \end{question}
  


\end{questions}
\end{document}

%%% Local Variables:
%%% mode: latex
%%% TeX-master: t
%%% TeX-engine: default
%%% End:
