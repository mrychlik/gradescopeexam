\documentclass[12pt]{gradescopeexam}
% \documentclass[12pt,noanswers,solutionkey]{gradescopeexam}
\noprintanswers
% \printanswers
% \printsolutionkey


\usepackage{amsmath}
\usepackage{amsthm}
\usepackage{bbm}
\usepackage{amsthm}
\usepackage{amsfonts}
\usepackage{times}
\usepackage{graphicx}
\usepackage{multicol}
\usepackage{amsthm}
\usepackage{datetime}
\usepackage{hyperref}
\usepackage{comment}
\usepackage{listings}
\usepackage{xcolor}
\usepackage{units}
\usepackage{MnSymbol}

\renewcommand\P{\mathbb{P}}
\newcommand\R{\mathbb{R}}
\newcommand\Q{\mathbb{Q}}
\newcommand\E{\mathbb{E}}
\newcommand\N{\mathbb{N}}
\newcommand\one{\mathbbm{1}}
\newcommand\V{\mathrm{var}}
\renewcommand\vec[1]{\mathbf{#1}}
\renewcommand\c[1]{\mathcal{#1}}
\newcommand\im[1]{\mathrm{Im}(#1)}

\newtheorem*{lemma}{Lemma}

\begin{document}

% ----------------------------------------------------------------
\LeftBoxMargin{0.0\linewidth}
\RightBoxMargin{0.0\linewidth}
\AssignmentTitle{Week4-template}
\CourseName{Math 563, Fall 2022}
% \FirstName{Marek}
% \LastName{Rychlik}
% \StudentId{31415926}
% ----------------------------------------------------------------
\makeheader
\vspace{0.1in}
\begin{questions}
  \begin{question}
    (Durrett 2.1.3.)
    Let $\rho(x, y)$ be a metric.
    \begin{parts}
      \part
      (i) Suppose $h$ is differentiable with $h(0) =0$, $h'(x) > 0$ for $x > 0$
      and $h'(x)$ decreasing on $[0, \infty)$. Then $h(\rho(x, y))$ is a metric
      \begin{prooforbox}[2.5in]
      \end{prooforbox}        
      \part $h(x)=x/(x+1)$ satisfies the hypothesis in (i)
      \begin{prooforbox}[2.5in]
      \end{prooforbox}
    \end{parts}
  \end{question}
  
  \begin{question}
    (Durrett 2.1.4.)
    Let $\Omega = (0, 1)$, $\c{F}=$ Borel sets, $\P =$ Lebesgue measure.
    $X_n (\omega) =\sin(2\pi n\omega)$, $n = 1, 2, \ldots$
    are uncorrelated but not independent.
    \begin{prooforbox}[6in]
    \end{prooforbox}
  \end{question}

  \begin{question}
    (Durrett 2.1.7.)  Let $K \ge 3$ be a prime and let $X$ and $Y$ be
    independent random variables that are uniformly distributed on
    $\{0, 1, \ldots , K -1\}$.  For $0 \le n < K$, let
    $Z_n = X + n\,Y \mod K$.
    \begin{parts}
      \part
      Show that $Z_0, Z_1, \ldots,Z_{K-1}$ are pairwise independent,
      i.e., each pair is independent.
      \begin{prooforbox}[3in]
      \end{prooforbox}
      \part
      They are not independent because if we know the values of two of
      the variables then we know the values of all the variables.

      {\color{magenta}
        (NOTE: Prove that the value of two determine the values of all)}
        
      \begin{prooforbox}[3in]
      \end{prooforbox}
    \end{parts}
  \end{question}

  \begin{question}
    (Durrett 2.1.14.)
    Let $X, Y \ge 0$
    be independent with distribution functions $F$ and
    $G$. Find the distribution function of $X\,Y$.
    \begin{solutionorbox}[6in]

    \end{solutionorbox}
  \end{question}
  
  \begin{question}
    (Durrett 2.1.15.)
    If we want an infinite sequence of coin tossings, we do not have
    to use Kolmogorov’s theorem. Let $\Omega$
    be the unit interval $(0,1)$ equipped
    with the Borel sets $\c{F}$ and Lebesgue measure $\P$.
    Let $Y_n (\omega) = 1$ if $\lfloor 2^n \omega\rfloor$
    is odd and $0$ if $\lfloor 2^n \omega\rfloor$ is even.
    Show that $Y_1 , Y_2 , \ldots$ are independent with
    $\P(Y_k = 0) = P (Y_k = 1) = 1/2$.
    \begin{prooforbox}[6in]
    \end{prooforbox}      
  \end{question}
\end{questions}
\end{document}

%%% Local Variables:
%%% mode: latex
%%% TeX-master: t
%%% TeX-engine: default
%%% End:
