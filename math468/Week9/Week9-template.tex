\documentclass[12pt]{gradescopeexam}
% \documentclass[12pt,noanswers,solutionkey]{gradescopeexam}
\noprintanswers
%\printanswers
%\printsolutionkey


\usepackage{amsmath}
\usepackage{bbm}
\usepackage{amsthm}
\usepackage{amsfonts}
\usepackage{times}
\usepackage{graphicx}
\usepackage{multicol}
\usepackage{amsthm}
\usepackage{datetime}
\usepackage{hyperref}
\usepackage{comment}
\usepackage{listings}
\usepackage{xcolor}
\usepackage{units}

\renewcommand\P{\mathbb{P}}
\newcommand\Q{\mathbb{Q}}
\newcommand\E{\mathbb{E}}
\newcommand\one{\mathbbm{1}}
\newcommand\V{\mathrm{var}}
\renewcommand\vec[1]{\mathbf{#1}}

\begin{document}

% ----------------------------------------------------------------
\LeftBoxMargin{0.0\linewidth}
\RightBoxMargin{0.0\linewidth}
\AssignmentTitle{Week9-template}
\CourseName{Math 468, Spring 2022}
\FirstName{Marek}
\LastName{Rychlik}
\StudentId{31415926}
% ----------------------------------------------------------------
\makeheader
\vspace{0.1in}
\begin{questions}
  \begin{question} (Durrett, Problem 4.1)
    A salesman flies around between Atlanta, Boston, and Chicago as the following rates (the units are trips per month):
    \begin{center}
      \begin{tabular}{|c|ccc|}
        \hline
        F\ T  & A & B & C\\
        \hline
        A &-4 & 2 & 2\\
        B & 3 &-4 & 1\\
        C & 5 & 0 &-5\\
        \hline
      \end{tabular}
    \end{center}

    \begin{parts}
      \part What is the transition rate matrix $\vec{Q}$ for this process?
      \begin{solutionorbox}[1.2in]
        \TheAnswer{
          \[ \vec{Q} =
            \begin{bmatrix}
              -4 &2 &2\\
              3 &-4 &1\\
              5 &0  &-5
            \end{bmatrix}
          \]
        }
      \end{solutionorbox}

      \part List the eigenvalues of $\vec{Q}$ as a comma-separated list.
      \begin{solutionorbox}[3.5in]
        \TheAnswer{$$-5,-8,0$$}
      \end{solutionorbox}

      \part Find the (right) diagonalizing matrix $\vec{S}$ of $\vec{Q}$, so
      that $\vec{S}^{-1}\vec{Q}\vec{S}$ is diagonal. Scale the columns so that
      the first entry in each column (counting from the top) is 1.
      \begin{solutionorbox}[3.5in]
        \TheAnswer{
          $$\vec{S}=
          \begin{bmatrix}
            0&1&1\\
            1&-{{1}\over{3}}&1\cr -1&-{{5}\over{3}}&1\\
          \end{bmatrix}
          $$}
      \end{solutionorbox}
      \part Find the left diagonalizing matrix $\vec{L}=\vec{S}^{-1}$ of $\vec{Q}$,
      so that $\vec{L}\vec{Q}\vec{L}^{-1}$ is diagonal.
      \begin{solutionorbox}[3.5in]
        \TheAnswer{
          $$\begin{bmatrix}
              -{{1}\over{3}}&{{2}\over{3}}&-{{1}\over{3}}\\
              {{1}\over{2}}&-{{1}\over{4}}&-{{1}\over{4}}\\
              {{1}\over{2}}&{{1}\over{4}}&{{1}\over{4}}
            \end{bmatrix}$$}
        \end{solutionorbox}
        \part What is the stationary distribution $\pi$ for this Markov process?
        {\color{magenta}Hint: Use one of the rows of the matrix found in the previous part.
          Make sure it is a row vector.}
        \begin{solutionorbox}[1.4in]    
          \TheAnswer{
            $$\pi=
            \begin{bmatrix}
              {{1}\over{2}}&{{1}\over{4}}&{{1}\over{4}}
            \end{bmatrix}$$
          }
        \end{solutionorbox}      
        \part (Durrett 4.1, part (a)) Find the limiting fraction of time she spends in each city.
        {\color{magenta} Only the exact answer yields credit. List the numbers in the order ``A, B, C''.}
        \begin{solutionorbox}[2in]
          \TheAnswer{$$1/2,1/4,1/4$$} 
        \end{solutionorbox}
        \part Find the routing matrix $\vec{R}$ for $\vec{Q}$.
        \begin{solutionorbox}[2.4in]
          \TheAnswer{$$\vec{R} =
            \begin{bmatrix}
              0 &1/2 &1/2\\
              3/4 &0 &1/4\\
              1 &0  &0
            \end{bmatrix}
            $$}
        \end{solutionorbox}
        \part If she is in Boston now, what is the probability that the first city she will visit next
        is Chicago?
        \begin{solutionorbox}[1.2in]
          \TheAnswer{$$\P(Y_1=C|Y_0=B)=\vec{R}_{23} = 1/4$$}
        \end{solutionorbox}
        \part (Durrett 4.1, part (b)) What is her average number of trips each year from Boston to Atlanta?
        \begin{solutionorbox}[6in]
          Once she is in Boston, she will visit Atlanta next according
          to the routing matrix $\vec{R}$, namely $\vec{R}_{21}=3/4$ of
          the times.  Hence, the answer is $3/4$ of the average number
          of visits to Boston per year.  As the fraction of time spend
          in Boston is $1/4$ of the time, and the rate of staying in
          Boston is $4$, the everage stay in Boston is $1/4$ (of a
          month).  Therefore, to stay $1/4$ of 1 year, which is $3$
          months spent in Boston each year, we must visit Boston $12$
          times a year. Clearly, $3/4$ of those visits is $9$ (trips to Atlanta).
          \TheAnswer{$$9$$}
        \end{solutionorbox}
        \part Find the matrix $\vec{P}(t)=e^{t\,\vec{Q}}$.
        \begin{solutionorbox}[7.8in]
          As we know, $\vec{Q}$ is diagonalizable and we know its diagonalizing matrix
          $$\vec{S}=
          \begin{bmatrix}
            0&1&1\\
            1&-{{1}\over{3}}&1\cr -1&-{{5}\over{3}}&1\\
          \end{bmatrix}
          $$
          and its inverse
          $$
          \vec{L}=
          \begin{bmatrix}
            -{{1}\over{3}}&{{2}\over{3}}&-{{1}\over{3}}\\
            {{1}\over{2}}&-{{1}\over{4}}&-{{1}\over{4}}\\
            {{1}\over{2}}&{{1}\over{4}}&{{1}\over{4}}
          \end{bmatrix}.
          $$
          Also, the diagonal form is
          $$
          \vec{D}=
          \begin{bmatrix}
            -5 &0 &0\\
            0 &-8 &0\\
            0 &0 &0\\
          \end{bmatrix}.
          $$
          Therefore,
          $$
          e^{t\,\vec{Q}}= \vec{S}e^{t\,\vec{D}}\vec{S}^{-1} =\vec{S}e^{t\,\vec{D}}\vec{L}.
          $$
          Hence the answer is:
          \begin{align*}
            e^{t\,\vec{Q}}
            &=
              \begin{bmatrix}
                0&1&1\\
                1&-\frac{1}{3}&1\\
                -1&-\frac{5}{3}&1\\
              \end{bmatrix}
            \begin{bmatrix}
              e^{-5t} &0 &0\\
              0 &e^{-8t} &0\\
              0 &0 &1\\
            \end{bmatrix}
            \begin{bmatrix}
              -\frac{1}{3} &\frac{2}{3} & -\frac{1}{3}\\
              \frac{1}{2}  &-\frac{1}{4}&-\frac{1}{4}\\
              \frac{1}{2}  &\frac{1}{4} &\frac{1}{4}
            \end{bmatrix}\\
            &=
              \begin{bmatrix}
                0&1&1\\
                1&-\frac{1}{3}&1\\
                -1&-\frac{5}{3}&1\\
              \end{bmatrix}
            \begin{bmatrix}
              -\frac{1}{3}e^{-5t}  &\frac{2}{3}e^{-5t}   & -\frac{1}{3}e^{-5t}\\
              \frac{1}{2}e^{-8t}   &-\frac{1}{4}e^{-8t}  &-\frac{1}{4}e^{-8t} \\
              \frac{1}{2}         &\frac{1}{4}         &\frac{1}{4}
            \end{bmatrix}\\
            &= \begin{bmatrix}
                 \frac{1}{2}\,e^{-8t} + \frac{1}{2} & -\frac{1}{4}\,e^{-8t} + \frac{1}{4} & -\frac{1}{4}\,e^{-8t} + \frac{1}{4}\\
                 -\frac{1}{3}\,e^{-5t} -\frac{1}{6}\,e^{-8t} + \frac{1}{2}  & \frac{2}{3}\,e^{-5t} +\frac{1}{12}\,e^{-8t} + \frac{1}{4}& -\frac{1}{3}\,e^{-5t}       +\frac{1}{12}\,e^{-8t} + \frac{1}{4} \\
                 \frac{1}{3}\,e^{-5t} -\frac{5}{6}\,e^{-8t} + \frac{1}{2}& -\frac{2}{3}\,e^{-5t} + \frac{5}{12}\,e^{-8t} + \frac{1}{4}& \frac{1}{3}\,e^{-5t} + \frac{5}{12}\,e^{-8t}+ \frac{1}{4}
               \end{bmatrix}
          \end{align*}
          
          \TheAnswer{
            \begin{small}
            $$\begin{bmatrix}{{e^ {- 8
                \,t }}\over{2}}+{{1}\over{2}}&{{1}\over{4}}-{{e^ {- 8\,t }}\over{4}}&{{1}\over{4}}-{{e^ {- 8\,t }}\over{4}}\\
                -{{e^ {- 5\,t }}\over{3}}-{{e^ {- 8\,t }}\over{6}}+{{1}\over{2}}&{{2\,e^ {- 5\,t }}\over{3}}+ {{e^ {- 8\,t }}\over{12}}+{{1}\over{4}}&-{{e^ {- 5\,t }}\over{3}}+{{e^ {- 8\,t }}\over{12}}+{{1}\over{4}}\\
                {{e^ {- 5\,t }}\over{3}}-{{5\,e^ {- 8\,t }}\over{6}}+{{1}\over{2}}&-{{2\,e^ {- 5\,t }}\over{3}}+{{5\,e^ {- 8\,t }}\over{12}}+{{1}\over{4}}&{{e^ {- 5\,t }}\over{3}}+{{5\,e^ {- 8\,t }}\over{12}}+{{1}\over{4}}\\ 
              \end{bmatrix} $$            
            \end{small}
          }
          \end{solutionorbox}
        \part If she is in Boston now, what is the probability that 
        she will be in Atlanta two months from now? {\color{magenta}Your answer must have at least 6 digits of precision.}
        \begin{solutionorbox}[2in]
          The answer is $\P(X(2)=A|X(0)=B)=\vec{P}_{21}(2)$ and can be obtained by evaluating the right entry of
          the matrix from the previous part.
          \TheAnswer{
            \begin{align*}
              \vec{P}_{21}(2) &=-{{e^ {- 5\,t }}\over{3}}-{{e^ {- 8\,t }}\over{6}}+{{1}\over{2}}\big|_{t=2}\\
                              &=-{{e^ {- 10 }}\over{3}}-{{e^ {- 16 }}\over{6}}+{{1}\over{2}}
                               \approx 0.4999848479342167
            \end{align*}
          }
        \end{solutionorbox}
      \end{parts}
    \end{question}
  \end{questions}
\end{document}
